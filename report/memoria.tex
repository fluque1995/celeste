\documentclass[12pt]{article} 
\usepackage[utf8x]{inputenc}
\usepackage{graphicx} 
\usepackage{multirow} 
\usepackage{hhline}
\usepackage{booktabs} 
\usepackage{vmargin} %cambia el margen
\usepackage{amsmath,amsthm} 
\usepackage{amsfonts} 
\usepackage{float}
\usepackage{listings}

\usepackage[hidelinks]{hyperref}

\title{
  Mecánica Celeste\\
  \large Trabajo de programación sobre la 
  órbita de los planetas en el sistema solar.  }


\author{ 
  Francisco Luque Sánchez
}

\begin{document}
\maketitle
\begin{center}  
\includegraphics[scale=0.35]{escudo.png}
\end{center}

\newpage

\tableofcontents % para generar el índice de contenidos

\pagebreak

\section{Introducción}

En esta práctica desarrollaremos un programa en Python que nos
permitirá conocer información sobre el estado de un planeta, visto
como una partícula en órbita en el Sistema Solar. Podremos calcular la
posición del mismo dependiendo del día del periodo orbital en el que
nos encontremos, así como su distancia al sol, la velocidad que tiene,
el módulo de la misma, su energía, su anomalía real y excéntrica, y su
momento angular.\\

Además, se nos permitirá representar en unos ejes cartesianos, en los
que se toma como punto de referencia el Sol, la órbita de los
planetas, así como su posición dentro de la órbita en un día
especificado por el usuario.\\

A lo largo de este informe vamos a explicar la fundamentación
matemática que justifica las funciones programadas en Python. El
código implementado se adjunta para su revisión en caso necesario.\\

Vamos ahora a ver la fundamentación matemática de las funciones
implementadas.

\section{Cálculos realizados}

Vamos a ver los cálculos realizados, que luego se han implementado
en funciones en Python para el programa.

\subsection{Posición de planeta}

Lo primero que se ha calculado es la posición del planeta. Para
calcular dicha posición, se ha utilizado el siguiente sistema:

\begin{equation}
  \left\{
    \begin{array}{l}
      x(t) = a(\cos{u(t)} - \varepsilon, \sqrt{1 - \varepsilon^2}\sin{u(t)}) \\
      u(t) - \varepsilon \sin{u(t)} = \frac{2\pi}{p}(t - t_0)      
    \end{array}
  \right.
\end{equation}

Lo primero que tenemos que calcular es la anomalía excéntrica. Hemos
supuesto $t_0 = 0$, ya que no tenemos información sobre un posible
punto de tiempo de referencia. Para poder calcular su valor, dado que
no podemos obtener una expresión analítica de la misma, vamos a
utilizar el método de Newton-Raphson para dar una aproximación
numérica.  Como ya vimos en las clases de teoría, definiendo
$\xi = \frac{2\pi}{p}$, y utilizando la expresión

\[
u_{n+1} = \frac{(-u_n\cos{u_n} + \sin{u_n})\varepsilon - \xi}{1 -
  \varepsilon \cos{u_n}}
\]

Tenemos una expresión iterativa basada en el método de Newton-Raphson
que nos permite aproximar el valor de la excentricidad en un tiempo
$t$.\\

Una vez calculado el valor de la anomalía excéntrica, simplemente 
sustituimos el valor en la ecuación de la posición, y obtenemos el
valor deseado.

\subsection{Distancia al Sol}

Para calcular la distancia al Sol, simplemente tendremos que calcular
el módulo de la posición calculada anteriormente.

\subsection{Velocidad del planeta}

Para calcular la velocidad, utilizaremos la derivada de la posición,
expresada de la misma manera que en apartado anterior. Tenemos que
dicha derivada es:

\[
\dot{x}(t) = a \dot{u}(t)(-\sin{u(t)}, \sqrt{1 -
  \varepsilon^2}\cos{u(t)})
\]

Y calculamos en clase que el valor de $\dot{u}(t)$ es:

\[
\dot{u}(t) = \frac{|c|}{a^2 \sqrt{1 - \varepsilon^2}(1 - \varepsilon
  \cos{u(t)})}
\]

Usando además la relación que nos liga el módulo del momento angular
con la longitud del semieje mayor de la elipse que define la órbita
del planeta, que es:

\[
a = \frac{|c|^2}{\mu(1 - \varepsilon^2)} \Rightarrow |c| =
\sqrt{a\mu(1 - \varepsilon^2)}
\]

Obtenemos que:

\[
\dot{u}(t) = \frac{\sqrt{a\mu}\sqrt{(1 - \varepsilon^2)}}{a^2 \sqrt{1
    - \varepsilon^2}(1 - \varepsilon \cos{u(t)})} =
\frac{\sqrt{a\mu}}{a^2 (1 - \varepsilon \cos{u(t)})}
\]

Ahora, con la tercera ley de Kepler podemos expresar $\mu$ en función
de otras constantes dependientes del planeta:

\[
p = \frac{2\pi}{\sqrt{\mu}}a^{\frac{3}{2}} \Rightarrow \sqrt{\mu} =
\frac{2\pi}{p}a^{3/2}
\]

Que sustituyendo en la ecuación, nos da el siguiente resultado:

\[
\dot{u}(t) = \frac{\sqrt{a\mu}}{a^2 (1 - \varepsilon \cos{u(t)})} =
\frac{2 \pi a^2}{a^2 p (1 - \varepsilon \cos{u(t)})} = \frac{2 \pi} {p
  (1 - \varepsilon \cos{u(t)})}
\]

Ahora sustituimos este valor en la expresión de $\dot{x}(t)$ para
obtener una expresión de la velocidad dependiente del tiempo:

\[
\dot{x}(t) = \frac{2 \pi a}{p(1 - \varepsilon
  \cos{u(t)})}(-\sin{u(t)}, \sqrt{1 - \varepsilon^2}\cos{u(t)})
\]

De nuevo, el cálculo de la anomalía excéntrica se hace con el método
de Newton-Raphson anteriormente explicado.

\subsection{Módulo de la velocidad}

Para obtener el módulo de la velocidad, calculamos la norma del
vector anterior.

\subsection{Anomalía real}

Además del sistema de ecuaciones anterior, la posición de un planeta
se puede expresar también con el siguiente sistema de ecuaciones:

\begin{equation}
  \left\{
    \begin{array}{l}
      x(t) = \frac{a(1 - \varepsilon^2)}{1 + \varepsilon(\cos{\theta(t)} - \omega)}(\cos{\theta(t)}, \sin{\theta(t)})\\
      \dot{\theta}(t) = \frac{|c|}{a^2(1 - \varepsilon^2)^2}(1 + \varepsilon \cos{(\theta(t) - \omega)})
    \end{array}
  \right.
\end{equation}

La función $\theta(t)$ se conoce como la anomalía real del planeta, y
marca el ángulo que forma, en radianes, el planeta con el semieje
mayor positivo de la elipse. Queremos averiguar dicho valor para un
tiempo $t$, pero no podemos obtener una expresión analítica de dicha
función, por lo que para resolverlo utilizaremos el método de
Runge-Kutta-4. Este método numérico sirve para estimar valores de una
función en un determinado punto $t$, conocida una expresión de la
derivada de la función a estimar (en nuestro caso una expresión de
$\dot{theta}(t)$, que tenemos escrita en el sistema anterior), y una
pareja $(t_0, \theta(t_0))$. Interpretando el significado de la
fúnción $\theta$, y suponiendo que $\omega = 0$, que indica el giro
del semieje mayor de la elipse respecto del eje $OX^{+}$, tenemos que
$\theta(t_0) = 0$, que será el punto de referencia que tomaremos
para utilizar el método seleccionado.\\

No haremos una explicación más profunda del funcionamiento de dicho
método, ya que se sale de los contenidos de la asignatura. Se puede
ver una explicación del método en
\url{https://en.wikipedia.org/wiki/Runge%E2%80%93Kutta_methods}

\subsection{Energía del planeta}

Pasamos ahora al cálculo de la energía del planeta. Por un lado,
tenemos que la energía de un cuerpo sometido a un campo de fuerzas
centrales en un determinado instante de tiempo $t$ es la siguiente:

\[
E_{t}(t) = E_{c}(t) + E_{p}(t) = \frac{m|\dot{x}(t)|^2}{2} -
\frac{m\mu}{|x(t)|}
\]

Donde nos aparecen involucrados el módulo de la velocidad del planeta,
anteriormente calculado, la distancia al Sol del mismo, cálculo que
también hemos realizado, y la constante $\mu$, para la que vimos antes
la expresión:

\[
\sqrt{\mu} = \frac{2\pi}{p}a^{3/2} \Rightarrow \mu = \frac{4
  \pi^2}{p^2}a^3
\]

Por lo que podemos calcular de forma sencilla el valor de la energía
en un determinado instante de tiempo $t$. Además, se dedujo, con la
ley de conservación de la energía, que la energía en todo momento es
constante. Podemos calcular entonces una expresión de la energía en
función de las constantes del planeta, que nos permita calcularla sin
tener que recurrir a la evaluación de la distancia al Sol y el módulo
de la velocidad en un instante de tiempo determinado. No mostramos
todos los cálculos intermedios dado que es un proceso tedioso, pero la
expresión de la energía la podemos calcular de la siguiente
manera. Tenemos por un lado que la distancia del planeta al Sol es,
tomando como posición el primer sistema que mostramos en la práctica:

\[
|x(t)| = a(1 - \varepsilon \cos{u(t)})
\]

Por otro, el cuadrado del módulo de la velocidad se puede expresar
como:

\[
|\dot{x}(t)|^2 = a^2|\dot{u}(t)|^2(1-\varepsilon^2cos^2{u(t)})
\]

Sustituyendo dichos valores en la expresión que pusimos al principio,
obtenemos:

\[
E_{t}(t) = \frac{ma^2|\dot{u}(t)|^2(1-\varepsilon^2cos^2{u(t)})}{2} -
\frac{m\mu}{a(1 - \varepsilon \cos{u(t)})}
\]

El valor del módulo de $|\dot{u}(t)|^2$ sabemos que es:

\[
|\dot{u}(t)|^2 = \frac{|c|^2}{a^4(1-\varepsilon^2)(1 - \varepsilon
  cos(u(t)))^2}
\]

Y conocemos la siguiente expresión para $\mu$:

\[
\mu = \frac{|c|^2}{a(1 - \varepsilon^2)}
\]

Al sustituir ambos valores en la expresión anterior y simplificar, se
obtiene que:

\[
E_t(t) = \frac{m|c|^2(1+\varepsilon
  cos(u(t)))}{2a^2(1-\varepsilon^2)(1-\varepsilon cos(u(t)))} -
\frac{m|c|^2}{a^2(1 - \varepsilon^2)(1-\varepsilon cos(u(t))}
\]

Convirtiendo las dos fracciones a común denominador, llegamos a la
siguiente expresión:

\[
E_t(t) = -\frac{m|c|^2}{2a^2(1-\varepsilon^2)}
\]

Que claramente es independiente del tiempo. El único valor que
tendríamos que calcular es el módulo del momento angular, que veremos
en la siguiente sección que es también constante. De esta forma,
tenemos una expresión constante para la energía, que en la
implementación compararemos con la energía en un instante de
tiempo, para comprobar la corrección de los cálculos.\\

Dado que no tenemos datos sobre la masa de los planetas, la
implementación de la energía que daremos será por unidad de masa. Para 
obtener el dato de la energía real de un planeta, habría que multiplicar
el valor que obtengamos por la masa del mismo.

\subsection{Momento angular del planeta}

\end{document}


